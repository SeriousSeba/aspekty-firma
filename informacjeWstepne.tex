\section{Informacje wstępne}
\subsection{Cel i okoliczności powstania}
Jesteśmy grupą studentów informatyki, więc zrozumiałe jest, że jesteśmy zafascynowani wszelkimi nowościami technologicznymi. Podczas jednego z naszych spotkań powstał pomysł, aby założyć firmę zajmującą się sprowadzaniem oraz sprzedażą najnowszych nowinek technologicznych od naszych przyjaciół z dalekiego wschodu. Naszym celem jest stworzenie przedsiębiorstwa, które uprości i przyspieszy zakupy chińskich produktów. W tej chwili bardzo dużo osób dokonuje takich zakupów na własną rękę martwiąc się o to kiedy lub czy w ogóle przesyłka dotrze albo zastanawiając się czy wygenerowane zostaną dodatkowe koszty dzięki wizycie w Urzędzie Celnym. Chcemy z tym skończyć. Nasza firma będzie sprowadzać sprzęt z Chin w możliwie najkorzystniejszych cenach oraz wysyłać i serwisować sprzęt na terenie naszego kraju. Aby to osiągnąć utworzymy nasz sklep internetowy, który będzie stanowił główne źródło pozyskiwania klientów. Ponieważ jesteśmy bardzo zdolnymi i ambitnymi studentami, stworzenie takiego sklepu nie będzie stanowiło najmniejszego problemu.
\subsection{Działalność przedsiębiorstwa}
Naszą główną działalnością będzie sprowadzanie, sprzedaż oraz serwis urządzeń elektronicznych z Chińskiej Republiki Ludowej. Obszarem, na którym będziemy działać jest terytorium naszego kraju. Docelową grupą klientów, którą będziemy chcieli pozyskać są młodzi ludzi, którzy zainteresowani są zakupem niekoniecznie markowych urządzeń, za to koniecznie w najniższej cenie. Naszą działalność chcemy promować głównie poprzez aktywne prowadzenie profili na portalach społecznościowych.
\subsection{Forma prawna}
Formą prawną, która została przez nas przyjęta jest spółka z ograniczoną odpowiedzialnością. W Polsce jest to druga najczęściej wybierana forma działalności gospodarczej. Może ona zostać utworzona przez jedną lub więcej osób, które nazywane są wspólnikami. Wspólnikami mogą być również inne podmioty posiadające osobowość prawną z jednym wyjątkiem. Jednoosobową spółkę z ograniczoną odpowiedzialnością nie może założyć inna jednoosobowa spółka z ograniczoną odpowiedzialnością. 

W Polsce spółka z o.o. jest spółką handlową regulowaną przez Kodeks spółek handlowych. Wspólnicy nie odpowiadają własnym majątkiem za zobowiązania wobec wierzycieli powstałe w wyniku prowadzonej działalności. Spółka odpowiada za nie całym swoim majątkiem. Jedynie w przypadku nieskutecznej egzekucji należności, członkowie zarządu mogą solidarnie odpowiadać za zobowiązania własnym majątkiem. Aby tego uniknąć członek zarządu powinien wykazać, że w odpowiednim momencie złożył wniosek o ogłoszenie upadłości. 

Do założenia spółki z ograniczoną odpowiedzialnością wymagana jest umowa w postaci aktu notarialnego. Można ją zastąpić umową w postaci elektronicznej podpisanej podpisem elektronicznym przez każdego ze wspólników. Minimalny kapitał zakładowy wynosi 5000 zł. Wartość nominalna udziału nie może być niższa niż 50 zł. Spółka jest również zobowiązana do prowadzenia ksiąg rachunkowych.

Organy spółki z ograniczoną odpowiedzialnością:
\begin{itemize}
	\item Zgromadzenie wspólników - najwyższa władza spółki. Podejmuje uchwały większością głosów. 
	\item Zarząd - powoływany jest przez zgromadzenie wspólników. Minimalnie jest to jedna osoba. Zarząd reprezentuje oraz prowadzi sprawy spółki. 
	\item Rada nadzorcza - jest obligatoryjna tylko w dwóch przypadkach: kapitał zakładowy osiągnął 500 000 zł i liczba wspólników jest większa od 25 osób lub spółka powstała ze spółki Skarbu Państwa. Zadaniem rady nadzorczej jest nadzór nad działalnością we wszystkich dziedzinach.
\end{itemize}

\subsection{Postępowanie przygotowawcze w celu założenia przedsiębiorstwa}
Przed przystąpieniem do kompletowania dokumentów niezbędnych do założenia spółki ustaliliśmy początkowe założenia. Ustaliśmy, że nazwą naszej firmy będzie: ChinaElectronics Spółka z o.o. z siedzibą: ul. Rozrywka 1, 30-001 Kraków. 