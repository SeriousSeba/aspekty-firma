\section{Rejestracja Firmy}
%wymienić w punktach i krótko opisać (w odpowiedniej kolejności) instytucje do których należy się udać, aby móc rozpocząć własną działalność gospodarczą (notariusz, urząd gminy, Krajowy Rejestr Sądowy, Urząd Statystyczny, Urząd Skarbowy, bank, ZUS etc.)\\
%jakie druki należy wypełnić w poszczególnych instytucjach?\\
%jakie dokumenty należy posiadać?\\
%wypełnić konieczne formularze i druki (wzory można ściągnąć internetu lub udać się do urzędu)\\

W celu założenia spółki z ograniczoną odpowiedzialnością należy odwiedzić szereg urzędów oraz instytucji. Poniżej przedstawiamy jak powinna przebiegać taka procedura.
\subsection{Notariusz}
Pierwszym przystankiem na naszej drodze do utworzenia spółki jest notariusz. W tym celu wszyscy wspólnicy muszą udać się do notariusza wraz ze swoimi dowodami osobistymi w celu podpisania umowy spółki z ograniczoną odpowiedzialnością. W takiej umowie uregulowane są wszystkie kwestie związane z przedmiotem działalności, organami spółki, kapitałem zakładowym oraz inne kwestie.  Przykład takiej umowy stanowi załącznik 1.
