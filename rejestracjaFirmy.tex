\section{Rejestracja Firmy}
%wymienić w punktach i krótko opisać (w odpowiedniej kolejności) instytucje do których należy się udać, aby móc rozpocząć własną działalność gospodarczą (notariusz, urząd gminy, Krajowy Rejestr Sądowy, Urząd Statystyczny, Urząd Skarbowy, bank, ZUS etc.)\\
%jakie druki należy wypełnić w poszczególnych instytucjach?\\
%jakie dokumenty należy posiadać?\\
%wypełnić konieczne formularze i druki (wzory można ściągnąć internetu lub udać się do urzędu)\\

W celu założenia spółki z ograniczoną odpowiedzialnością należy odwiedzić szereg urzędów oraz instytucji. Poniżej przedstawiamy jak powinna przebiegać taka procedura.
\subsection{Notariusz}
Pierwszym przystankiem na naszej drodze do utworzenia spółki jest notariusz. W tym celu wszyscy wspólnicy muszą udać się do notariusza wraz ze swoimi dowodami osobistymi w celu podpisania umowy spółki z ograniczoną odpowiedzialnością. W takiej umowie uregulowane są wszystkie kwestie związane z przedmiotem działalności, organami spółki, kapitałem zakładowym oraz inne kwestie.  Przykład takiej umowy stanowi załącznik 1.

\subsection{Utworzenie kapitału zakładowego}
Kapitał zakładowy jest niezbędny do dalszego przeprowadzenia procesu zakładania spółki. Każdy ze wspólników musi złożyć zadeklarowaną w umowie sumę. Ponieważ na tym etapie spółka nie posiada jeszcze konta bankowego, kapitał zakładowy może zostać złożony np. u prezesa zarządu, który po założeniu konta bankowego umieści go tam.

\subsection{KRS}
Krajowy Rejestr Sądowy jest instytucją, do której musimy się skierować po załatwieniu formalności u notariusza i utworzeniu kapitału zakładowego. KRS składa się z rejestrów: przedsiębiorców, stowarzyszeń i innych organizacji społecznych i zawodowych oraz dłużników niewypłacalnych. Nas na tę chwilę interesuje rejestr przedsiębiorców. To tam po załatwieniu formalności umieszczone zostaną informacje na temat naszej firmy. Przed wizytą w urzędzie należy się dobrze przygotować, wypełnić wszystkie wymagane dokumenty, zebrać oświadczenia oraz załączniki. 

Krajowy Rejestr Sądowy wymaga od nas przygotowania formularzy:
\begin{enumerate}
	\item Wniosek o rejestrację spółki z ograniczoną odpowiedzialnością (KRS-W3) 
	\item Załącznik do wniosku o rejestrację podmiotu w rejestrze przedsiębiorców – wspólnicy spółki z ograniczoną odpowiedzialnością podlegający wpisowi do rejestru (KRS-WE)
	\item Załącznik do wniosku o rejestrację lub zmianę danych – organy podmiotu lub wspólnicy uprawnieni do reprezentowania spółki (KRS-WK)
	\item Załącznik do wniosku o rejestrację podmiotu – prokurenci (o ile są powoływani) (KRS-WL)
	\item Załącznik do wniosku o rejestrację podmiotu w rejestrze przedsiębiorców – przedmiot działalności (KRS-WM)
\end{enumerate}

Dodatkowo w ramach procedury "jednego okienka" należy przygotować i złożyć razem z wnioskiem następujące załączniki:

\begin{enumerate}
	\item Umowa spółki z o.o. - umowa, która została przygotowana u notariusza. W przypadku wersi papierowej musi być sporządzona w formie aktu notarialnego. Jeżeli rejestracja przebiega przez internet wystarczy umowa podpisana przez wszystkich wspólników za pomocą elektronicznego podpisu.
	\item Oświadczenie wszystkich wspólników na temat wniesienia wkładu stanowiącego kapitał zakładowy.
	\item W przypadku braku informacji na tema zarządu spółki w umowie spółki z o.o. należy również przedstawić dowód ustanowienia tych organów wraz z danymi osobowymi wybranych osób.
	\item Lista wspólników podpisana przez wszystkich członków zarządu
	\item Złożone wobec sądu lub poświadczone notarialnie wzory podpisów każdego członka zarządu.
	\item Dowód uiszczenia opłaty sądowej oraz opłaty za ogłoszenie w Monitorze Sądowym i Gospodarczym 
\end{enumerate}

Komplet dokumentów należy złożyć w sądzie rejonowym właściwym ze względu na siedzibę spółki. Pozostało jeszcze wspomnieć o opłatach za rejestrację. Opłata sądowa za wpis wynosi 500 zł. Do tego trzeba doliczyć 100 zł opłaty za zamieszczenie w Monitorze Sądowym i Gospodarczym ogłoszenia o wpisie do KRS.
Na tym etapie nie trzeba zgłaszać się do Zakładu Ubezpieczeń Społecznych oraz do Urzędu Statystycznego w celu otrzymania odpowiednio numeru NIP oraz REGON. Numery te są teraz nadawane automatycznie. Otrzymamy je razem z postanowieniem sądu o wpisie do KRS.

\subsection{Bank}
Teoretycznie nie ma obowiązku posiadania konta bankowego przez spółki z o.o. Jednak nie wyobrażamy sobie funkcjonowania bez posiadania takiego konta.

Do banku możemy się udać dopiero w momencie otrzymania decyzji o wpisie do KRS. Z tą decyzją udajemy się do banku w celu założenia konta. 

Posiadając już komplet dokumentów warto również zaopatrzyć się w pieczęć. Również nie jest to wymagane, jednak jest to bardzo przydatny dodatek w trakcie prowadzenia działalności.

\subsection{ZUS oraz GUS}
Jak już powyżej wspomnieliśmy ZUS oraz GUS otrzymują informację na temat naszej spółki automatycznie, podczas składania wniosku do KRS. Numer NIP oraz REGON otrzymamy wraz z decyzją o wpisie do KRS. Z tego powodu nie musimy odwiedzać osobiście tych urzędów.

Należy jednak poinformować te urzędy o numerze konta bankowego spółki (lub jego braku). Informację taką należy zgłosić do 7 dni (w przypadku ZUS) lub 21 dni (w przypadku GUS). W praktyce oznacza to wypełnienie kolejnego formularza. Należy uzupełnić NIP-8 i złożyć go we właściwym ze względu na siedzibę urzędzie skarbowym.
 
\subsection{Podsumowanie}
Jak widać większość formalności załatwiamy po jednej wizycie w KRS. Dzięki znacznemu uproszczeniu procedur związanych z zakładaniem spółki omija nas trud kompletowania wielu dodatkowych dokumentów, które do niedawna były wymagane (np. wniosek do ZUS, GUS, US, dokumenty potwierdzające prawo do lokalu, drugi egzemplarz umowy). Po wykonaniu powyższych procedur możemy zakasać rękawy i rozpocząć prowadzenie własnej działalności.

